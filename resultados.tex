\documentclass[12pt]{article}
\usepackage[utf8]{inputenc}
\usepackage[spanish]{babel}
\usepackage{amsmath}
\usepackage{graphicx}
\usepackage{geometry}
\usepackage{hyperref}
\geometry{margin=2.5cm}
\title{Análisis de Resultados: Resolución de la Coalición Ganadora Mínima (CGM)}
\author{Basado en Lincolao-Venegas et al. (2023)}
\date{Julio 2025}

\begin{document}

\maketitle

\section*{Contexto del Problema}

El objetivo es encontrar una \textbf{coalición ganadora mínima} (MWC o CGM) de congresistas, es decir, un subconjunto de tamaño $q$ (quórum requerido) tal que la \emph{suma total de las distancias políticas entre sus miembros sea mínima}. Estas distancias se calculan como distancias euclídeas en el espacio político bidimensional obtenido mediante DW-NOMINATE.

\section*{Caso de Estudio: Votación RH0941234 (75º Congreso de EE.UU.)}

\begin{itemize}
    \item \textbf{Congresistas:} 431 representantes.
    \item \textbf{Quórum absoluto requerido:} $q = 216$
    \item \textbf{Datos:} Posiciones DW-NOMINATE descargadas desde \texttt{voteview.com}
    \item \textbf{Motivo de elección:} Bajo nivel de polarización política, ideal para validar la heurística.
\end{itemize}

\section*{1. Construcción de Coaliciones Iniciales}

Para cada congresista $p_i$, se generó una coalición $G_i$ formada por los 215 congresistas más cercanos a $p_i$ (según distancia euclídea) junto con él mismo, resultando en 431 coaliciones iniciales.

\begin{itemize}
    \item Se evaluó el \textbf{fitness} $Z(G_i)$ como la suma de todas las distancias euclídeas entre los pares de miembros de la coalición.
    \item Se eligió como mejor candidato inicial la coalición centrada en \textbf{Tim Lee Hall} (Demócrata, Illinois) con:
    \[
    Z(G_c) \approx 9689.948859
    \]
    \item Composición: Mayoritariamente Demócratas, con solo 15 Republicanos.
\end{itemize}

\section*{2. Mejora por Búsqueda Local y Polígono Convexo}

Se aplicó una heurística de búsqueda local sobre el \textbf{polígono convexo} de la coalición inicial $G_c$:

\begin{enumerate}
    \item Construcción del polígono convexo $H$ de $G_c$ (usando el algoritmo de Graham).
    \item Cálculo del centroide $C$.
    \item Ordenar los vértices de $H$ según su distancia a $C$ (de mayor a menor).
    \item Buscar reemplazos en el conjunto de congresistas fuera de $G_c$ que estén dentro de un radio proporcional al vértice más alejado ($\alpha \cdot r$).
    \item Si se encuentra un reemplazo que mejora el fitness, se actualiza la coalición y se repite el proceso.
\end{enumerate}

\section*{3. Iteraciones del Algoritmo de Mejora}

\subsection*{Primera Iteración}
\begin{itemize}
    \item Se reemplazó a \textbf{Gillis W. Long} (Demócrata del Sur) por \textbf{Albert W. Johnson} (Republicano, Pennsylvania).
    \item Nuevo fitness:
    \[
    Z(G_c) = 9688.493662454
    \]
\end{itemize}

\subsection*{Segunda Iteración}
\begin{itemize}
    \item Se reemplazó a \textbf{Henry B. González} (Demócrata, Texas) por \textbf{Matthew J. Rinaldo} (Republicano, New Jersey).
    \item Nuevo fitness:
    \[
    Z(G_c) = 9686.938309982
    \]
    Coincide con el valor alcanzado por el algoritmo genético.
\end{itemize}

\subsection*{Tercera Iteración}
\begin{itemize}
    \item No se encontraron reemplazos que mejoraran la coalición. Se alcanzó convergencia.
\end{itemize}

\subsection*{Coalición Final}
\begin{itemize}
    \item 199 Demócratas
    \item 17 Republicanos
    \item Fitness final:
    \[
    Z(G^*) = 9686.938309982
    \]
\end{itemize}

\section*{4. Comparación con el Algoritmo Genético (GA)}

\begin{itemize}
    \item \textbf{Tamaño de población:} 38
    \item \textbf{Tasa de mutación:} 0.1700019
    \item \textbf{Tasa de selección:} 0.141
    \item \textbf{Iteraciones promedio:} 15\,315
    \item \textbf{Tiempo promedio:} 26.5 segundos
    \item \textbf{Fitness óptimo alcanzado:}
    \[
    Z(G^*) = 9686.93831 \quad (\text{91.41\% de las ejecuciones})
    \]
\end{itemize}

\textbf{Comparación:} El algoritmo ad-hoc fue $\sim$265 veces más rápido (0.1 segundos) y alcanzó el mismo valor óptimo que el GA.

\section*{5. Observaciones Técnicas}

\begin{itemize}
    \item El uso del \textbf{polígono convexo} permitió identificar los miembros más extremos y sustituirlos por congresistas ideológicamente más cercanos al centroide.
    \item La mejora del fitness fue guiada por la estructura geométrica del espacio político.
    \item El algoritmo convergió en 2 iteraciones de mejora, validando la eficacia de la heurística.
\end{itemize}

\section*{6. Implementación}

\begin{itemize}
    \item Lenguaje: C++
    \item Compiladores: \texttt{CMake 3.22.1}, \texttt{GCC 7.5.0}
    \item Sistema operativo: openSUSE Leap 15.2
    \item Plataforma: Servidor Lenovo ThinkSystem SR530, 2x Intel Xeon Gold 5220
    \item Código fuente disponible: \url{https://github.com/vicquiroz/MWC.git}
\end{itemize}

\section*{Conclusión}

La Sección VII del artículo demuestra que el problema de la CGM puede resolverse eficazmente con un algoritmo determinista basado en construcción heurística y mejora local. La convergencia rápida y la coincidencia con la solución del algoritmo genético validan su efectividad tanto en precisión como en eficiencia computacional.

\end{document}