\documentclass[12pt]{article}
\usepackage[utf8]{inputenc}
\usepackage{amsmath, amssymb}
\usepackage{graphicx}
\usepackage{geometry}
\usepackage{hyperref}
\geometry{margin=1in}

\title{Resumen Detallado del Artículo: \\
\textit{An ad-hoc algorithm to find the minimum winning coalition}}
\author{}
\date{}

\begin{document}

\maketitle

\section*{III. Definición del problema}

El problema abordado es una variante del agrupamiento con restricciones de tamaño, similar al Min-Sum 2-Clustering Euclídeo Cuadrático. El objetivo es identificar una Coalición Ganadora Mínima (CGM), un subconjunto de congresistas cuya suma de distancias euclidianas entre sí sea mínima.

Se define el conjunto de congresistas del parlamento como:

\begin{equation}
P = \{p_1, p_2, p_3, ..., p_n\}
\end{equation}

donde cada congresista \( p_i = (X_i, Y_i) \) representa sus coordenadas políticas (según DW-Nominate).

La distancia entre dos congresistas se mide mediante la distancia euclidiana:

\[
d(p_i, p_j) = \sqrt{(X_i - X_j)^2 + (Y_i - Y_j)^2}
\]

El número de coaliciones posibles de tamaño \( q \) es:

\begin{equation}
C(n, q) = \binom{n}{q} = \frac{n!}{q!(n - q)!}
\end{equation}

Buscamos un subconjunto \( G \subset P \) tal que:

\begin{equation}
G = \{g_1, g_2, ..., g_q\}
\end{equation}

minimizando la función objetivo:

\begin{equation}
Z(G) = \sum_{i=1}^{q-1} \sum_{j=i+1}^{q} d(g_i, g_j)
\end{equation}

La CGM es entonces:

\begin{equation}
G^* = \arg\min_G Z(G)
\end{equation}

\section*{IV. Solución del problema}

Se propone un algoritmo greedy para encontrar una solución inicial \( G_c \). Para cada congresista \( p_i \in P \), se forma una coalición \( G_i \) con los \( q-1 \) más cercanos a él:

\begin{equation}
G_i = \{g_i, gr_1, ..., gr_{q-1}\}
\end{equation}

\[
|G_i| = q
\]

Se define:

\begin{equation}
G = \{(G_i, Z(G_i)) : i = 1, ..., n\}
\end{equation}

Se escoge como solución inicial:

\begin{equation}
G_c = \arg\min_{G_i} Z(G_i)
\end{equation}

Posteriormente, se aplica un algoritmo de búsqueda local que reemplaza miembros del polígono convexo de \( G_c \), usando un radio dinámico \( r = \alpha \cdot r_{\text{max}} \), donde \( \alpha > 1 \), y selecciona puntos candidatos fuera del polígono para mejorar el fitness.

\section*{Algorithm 1: Algoritmo de búsqueda local}

\textbf{Entrada:} \( P, G_c, \alpha \) \\
\textbf{Salida:} \( G^* \)

\begin{enumerate}
    \item \( G^* \gets G_c \)
    \item \( C \gets \text{Centroide}(G_c) \)
    \item Repetir:
    \begin{enumerate}
        \item \( G^*_{\text{orig}} \gets G^* \)
        \item \( H \gets \text{Polígono convexo de } G_c \)
        \item \( H' \gets H \text{ ordenado decrecientemente según distancia a } C \)
        \item \( r \gets \alpha \cdot \max_{h_j \in H'} d(h_j, C) \)
        \item \( P_c \gets \text{Puntos fuera de } G^* \text{ dentro de radio } r \)
        \item Para cada \( h_j \in H' \), reemplazar por cada \( p_k \in P_c \), calcular nuevo \( Z(G_c^k) \)
        \item Escoger \( G_c = \arg\min_k Z(G_c^k) \)
        \item Si \( Z(G_c) < Z(G^*) \), actualizar \( G^* \gets G_c \), recalcular centroide y \( H' \)
    \end{enumerate}
    \item Hasta que \( G^* = G^*_{\text{orig}} \)
\end{enumerate}

\section*{V. Caso de estudio}

Se aplicó el algoritmo al 75° Congreso de los Estados Unidos (1975–1977), con 431 congresistas activos y \( q = 216 \).

\subsection*{Razones de elección:}
\begin{itemize}
    \item Número elevado de participantes, lo que impide el uso de algoritmos deterministas clásicos como branch-and-bound.
    \item Baja polarización, lo que permite estudiar CGM bipartidistas.
\end{itemize}

\subsection*{Resultados:}
\begin{itemize}
    \item La solución inicial \( G_c \): \( Z(G_c) = 9689.95 \), principalmente demócratas.
    \item Iteración 1: Se reemplaza Long (Demócrata) por Johnson (Republicano) → \( Z = 9688.49 \)
    \item Iteración 2: Se reemplaza González (Demócrata) por Rinaldo (Republicano) → \( Z = 9686.93 \)
    \item Iteración 3: No se encontraron mejoras → convergencia alcanzada.
\end{itemize}

\subsection*{Comparación con algoritmo genético:}
\begin{itemize}
    \item Ambos algoritmos llegan al mismo valor óptimo: \( Z = 9686.93 \)
    \item Tiempo del algoritmo propuesto: \( \approx 0.1 \) segundos.
    \item Tiempo del algoritmo genético: \( \approx 26.5 \pm 4.1 \) segundos.
    \item Mejora de dos órdenes de magnitud en eficiencia temporal.
\end{itemize}

\section*{V. Caso de estudio}

Para validar la efectividad del algoritmo, se utilizó como caso de estudio la Cámara de Representantes del 75° Congreso de los Estados Unidos de América (1975–1977), conformada por 435 miembros. De estos, participaron activamente 431 en la votación seleccionada, lo que fija el quórum mínimo necesario para la mayoría absoluta en:

\[
q = 216
\]

\subsection*{Criterios de selección}

Se eligió este caso por dos razones principales:
\begin{enumerate}
    \item \textbf{Tamaño del cuerpo legislativo:} La alta cantidad de representantes hace inviable el uso de algoritmos deterministas tradicionales como \textit{branch-and-bound} o fuerza bruta, incluso con herramientas modernas como CPLEX, Gurobi o SCIP.
    \item \textbf{Baja polarización política:} Según estudios históricos, el 75° Congreso fue uno de los menos polarizados desde 1970, ideal para evaluar coaliciones multipartidistas. El 66\% de los representantes pertenecía al Partido Demócrata, y el 34\% al Partido Republicano.
\end{enumerate}

\subsection*{Datos utilizados}

Se utilizó la votación identificada como RH0941234 (Proyecto de Ley HRE1553), cuyos datos están disponibles en \href{https://voteview.com/}{Voteview.com}, una base de datos con las coordenadas DW-Nominate de los representantes para cada votación. Este sistema representa las posturas políticas en un espacio bidimensional: la primera dimensión suele estar relacionada con temas económicos y la segunda con temas sociales.

\subsection*{Observaciones iniciales}

En la votación seleccionada:
\begin{itemize}
    \item La dispersión de los representantes demócratas en la primera dimensión era baja, con excepción de Lawrence Patton McDonald, un miembro de la extrema derecha demócrata (``Dixiecrat'').
    \item En la segunda dimensión había alta variabilidad entre demócratas, lo que indicaba divisiones internas sobre temas sociales y culturales.
    \item Esto sugiere la posibilidad de que una CGM bipartidista centrada sea más estable que una unipartidista, especialmente en votaciones sensibles.
\end{itemize}

\subsection*{Ejecución del algoritmo}

\paragraph{Solución inicial:}
El algoritmo determinista seleccionó como punto de partida la coalición centrada en Tim Lee Hall (Demócrata de Illinois). Esta coalición inicial incluía a 201 demócratas y 15 republicanos:

\[
Z(G_c) = 9689.95
\]

\paragraph{Primera iteración:}
El algoritmo detectó que el representante más alejado del centroide era Gillis William Long (Demócrata de Louisiana). Fue reemplazado por Albert W. Johnson (Republicano de Pennsylvania), con lo cual se mejoró el fitness:

\[
Z = 9688.49
\]

\paragraph{Segunda iteración:}
Se reemplazó a Henry B. González (Demócrata de Texas) por Matthew John Rinaldo (Republicano de New Jersey), lo que llevó al mismo valor óptimo obtenido por el algoritmo genético:

\[
Z(G^*) = 9686.93
\]

\paragraph{Tercera iteración:}
Con un nuevo radio \( r = 0.4971 \), no se encontró ningún candidato que mejorara la coalición actual. El algoritmo concluyó que se había alcanzado la CGM.

\subsection*{Análisis de resultados}

\begin{itemize}
    \item La CGM final incluye 199 demócratas y 17 republicanos.
    \item En todas las iteraciones, se observa una sustitución de demócratas del sur por republicanos del este, reflejando diferencias ideológicas regionales.
    \item Se logró la misma solución que el algoritmo genético, pero en tan solo $\approx 0.1$ segundos, versus $\approx 26.5 \pm 4.1$ segundos del genético.
\end{itemize}

\subsection*{Conclusión}

Este estudio muestra que el algoritmo propuesto:
\begin{itemize}
    \item Encuentra soluciones de alta calidad equivalentes a métodos metaheurísticos.
    \item Reduce significativamente el tiempo de ejecución (tres órdenes de magnitud).
    \item Es adecuado para cuerpos legislativos grandes y moderadamente polarizados.
\end{itemize}

\end{document}